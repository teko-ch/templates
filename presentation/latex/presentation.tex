% !TEX TS-program = xelatex
% !TEX encoding = UTF-8 Unicode
% !BIB TS-program = biber

\documentclass{beamer}
\usepackage[cm-default,no-math]{fontspec}
\usepackage{xunicode}
\usepackage{xltxtra}
\usepackage[utf8x]{inputenc}
\usepackage{listings}
\usepackage{hyperref} 
\usepackage{adjustbox}
\usepackage{tcolorbox}
\usepackage{pdfpages}
\usepackage{multicol}
\usepackage{cancel}
\usepackage[autostyle=true,german=quotes]{csquotes}
\usepackage{pifont}
\usepackage{color}
\usepackage{qrcode}

\usetheme{m} % Use metropolis theme


\definecolor{pblue}{rgb}{0.13,0.13,1}
\definecolor{pgreen}{rgb}{0,0.5,0}
\definecolor{pred}{rgb}{0.9,0,0}
\definecolor{pgrey}{rgb}{0.46,0.45,0.48}

\usepackage{listings}
\lstset{language=Java,
  frame=single,
  showspaces=false,
  showtabs=false,
  breaklines=true,
  numbers=left,
  showstringspaces=false,
  breakatwhitespace=true,
  commentstyle=\color{pgreen},
  keywordstyle=\color{pblue},
  tabsize=4,
  xleftmargin=8pt,
  stringstyle=\color{pred},
  basicstyle=\footnotesize\ttfamily,
  moredelim=[il][\textcolor{pgrey}]{$$},
  moredelim=[is][\textcolor{pgrey}]{\%\%}{\%\%}
}

\setbeamertemplate{itemize items}[square]
\setbeamercovered{transparent}

%% Commands

\newcommand{\code}[1]{\texttt{#1}}

\newcommand{\listing}[1]{
	\begin{itemize}
		\item[]\lstinputlisting[]{listings/#1}
	\end{itemize}
}

\graphicspath{ {./images/} }
\newcommand{\myfig}[2]{
	\begin{minipage}[c]{\textwidth}
		\begin{center}
			\includegraphics[keepaspectratio,width=#2\textwidth]{#1}
		\end{center}
		\vspace{3mm}
	\end{minipage}
}

\newcommand{\bb}[1]{\textbf{#1}}

\newcommand{\slideItems}[1]{
	\begin{itemize}
		#1
	\end{itemize}
}

\newcommand{\slide}[2]{
	\begin{frame}{#1}
		#2
	\end{frame}
}

%% Document

\title{Course Title}
\subtitle{Subtitle of presentation}
\author{\href{http://www.teko.ch/}{Your Name}}
\institute[TEKO]{\href{http://www.teko.ch/}{TEKO Schweizerische Fachschule}}
\date{\today}

\begin{document}

\maketitle
\newlength\someheight

\slide{Agenda}{
	\setcounter{tocdepth}{1}
	\tableofcontents
}

\section{Lists}

\slide{Simple List}{
	\slideItems{
		\item This is item 1
		\item This is item 2
		\item This is item 3
		\item This is item 4
		\item This is item 5
	}
}

\slide{Nested List}{
	\slideItems{
		\item This is item 1
		\item This is item 2
		\item This is item 3
			\slideItems{
				\item This is item 3.1
				\item This is item 3.2
				\item This is item 3.3
			}
		\item This is item 4
			\slideItems{
				\item This is item 4.1
				\item This is item 4.2
			}
	}
}

\section{Links}

\slide{Clickable URL}{
	You can find the TEKO website here: \url{http://www.teko.ch/}
}

\slide{Clickable Text}{
	You should go to the \href{http://www.teko.ch/}{TEKO website} and check for courses!
}

\section{Text Size}

\slide{Define font size for one or more words}{
	\tiny{tiny}
	\scriptsize{scriptsize}
	\footnotesize{footnotesize}
	\small{small}
	\normalsize{normalsize} \\
	\large{large}
	\Large{Large}
	\LARGE{LARGE}
	\huge{huge}
	\Huge{Huge}
}

\slide{Define font size for a block}{
	\begin{LARGE}
		This block is in a large font.
	\end{LARGE}
}

\section{Images}

\slide{Image}{
	\myfig{cat.jpg}{1}
	\tiny{\href{https://www.flickr.com/photos/pagedooley/3372925208}{Image by Kevin Dooley}}
}

\section{Code}

\slide{Inline Code}{
	\slideItems{
		\item This is \code{item} 1
		\item This is \code{item} 2
		\item This is \code{item} 3
		\item This is \code{item} 4
		\item This is \code{item} 5
	}
}

\slide{Import Listing}{
	\listing{Listing.java}
}

\section{Tables}

\slide{Simple Table}{
	\begin{center}
		\begin{tabular}{ l || c | c | c | c | c }
			& 1 & 2 & 3 & 4 & 5 \\
			\hline
			\hline
			A & A1 & A2 & A3 & A4 & A5 \\
			B & B1 & B2 & B3 & B4 & B5 \\
			C & C1 & C2 & C3 & C4 & C5 \\
			D & D1 & D2 & D3 & D4 & D5 \\
			E & E1 & E2 & E3 & E4 & E5 \\
			\hline
			\hline
		\end{tabular}
	\end{center}
}

\section{Symbols}

\slide{Using Zapf Dingbats}{
	\ding{33} \ding{34} \ding{35} \ding{36} \ding{37} \ding{38} \ding{39}
	\ding{40} \ding{41} \ding{42} \ding{43} \ding{44} \ding{45} \ding{46} \ding{47} \ding{48} \ding{49}
	\ding{50} \ding{51} \ding{52} \ding{53} \ding{54} \ding{55} \ding{56} \ding{57} \ding{58} \ding{59}
	\ding{60} \ding{61} \ding{62} \ding{63} \ding{64} \ding{65} \ding{66} \ding{67} \ding{68} \ding{69}
	\ding{70} \ding{71} \ding{72} \ding{73} \ding{74} \ding{75} \ding{76} \ding{77} \ding{78} \ding{79}
	\ding{80} \ding{81} \ding{82} \ding{83} \ding{84} \ding{85} \ding{86} \ding{87} \ding{88} \ding{89}
	\ding{90} \ding{91} \ding{92} \ding{93} \ding{94} \ding{95} \ding{96} \ding{97} \ding{98} \ding{99}
	
	For more symbols look here: \url{https://www.rpi.edu/dept/arc/training/latex/LaTeX_symbols.pdf}
}

\section{Colors}

\slide{Colors}{
	\textcolor{green}{This text is green.} \\
	\textcolor{yellow}{This text is yellow.} \\
	\textcolor{red}{This text is red.}
}

\section{Columns}

\slide{Two Columns (50/50)}{
	\begin{columns}
		\begin{column}{0.5\textwidth}
			\myfig{cat.jpg}{1}
		\end{column}
		\begin{column}{0.5\textwidth}
			Two columns (50/50) with an image on the left side and a text on the right side.
		\end{column}
	\end{columns}
	\tiny{\href{https://www.flickr.com/photos/pagedooley/3372925208}{Image by Kevin Dooley}}
}

\slide{Two Columns (70/30)}{
	\begin{columns}
		\begin{column}{0.7\textwidth}
			\myfig{cat.jpg}{1}
		\end{column}
		\begin{column}{0.3\textwidth}
			Two columns (70/30) with an image on the left side and a text on the right side.
		\end{column}
	\end{columns}
	\tiny{\href{https://www.flickr.com/photos/pagedooley/3372925208}{Image by Kevin Dooley}}
}

\slide{Three Columns (20/50/30)}{
	\begin{columns}
		\begin{column}{0.2\textwidth}
			\myfig{cat.jpg}{1}
		\end{column}
		\begin{column}{0.5\textwidth}
			Three columns (20/50/30) with an image on the left side, a text in the middle and an image on the right side.
		\end{column}
		\begin{column}{0.3\textwidth}
			\myfig{cat.jpg}{1}
		\end{column}
	\end{columns}
	\tiny{\href{https://www.flickr.com/photos/pagedooley/3372925208}{Image by Kevin Dooley}}
}

\section{QR-Code}

\slide{QR-Code}{
	\begin{center}
		\qrcode[hyperlink,height=5cm]{http://www.teko.ch/}
	\end{center}
}

\end{document}
